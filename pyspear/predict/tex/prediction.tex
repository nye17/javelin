\documentclass[12pt,letterpaper]{article}

\usepackage{graphicx}
\usepackage{wrapfig}
\usepackage{grffile}
\usepackage{natbib}
\usepackage{amsmath}	% just math
\usepackage{amssymb}	% allow blackboard bold (aka N,R,Q sets)
\usepackage{amsthm}	% allows thm environment

%\linespread{1.8}	% double spaces lines
\linespread{0.9}	% single spaces lines

\textwidth 6.5truein  % These 4 commands define more efficient margins
\textheight 9.5truein
\oddsidemargin 0.0in
\topmargin -0.6in

\parindent 0pt	% let's not indent paragraphs
\parskip 5pt  % Also, a bit of space between paragraphs

\def\ok{yeah}
\def\sstarhat{\widehat{s_*}}
\def\sstar{s_*}
\def\ustar{u_*}
\def\bcsstar{\mathbf{S}_*}
\def\balpha{\mathbf{\alpha}}
\def\bv{\mathbf{v}}
\def\by{\mathbf{y}}
\def\bs{\mathbf{s}}
\def\bn{\mathbf{n}}
\def\br{\mathbf{r}}
\def\bx{\mathbf{x}}
\def\bcs{\mathbf{S}}
\def\bcn{\mathbf{N}}
\def\bcc{\mathbf{C}}
\def\bcq{\mathbf{Q}}

\begin{document}
\begin{flushright}
\linespread{1}	% single spaces lines
\small \normalsize %% dumb, but have to do this for the prev to work
Predictive Distribution of Light Curves as Gaussian Processes\\
YZ \\
Last Updated: \today\footnote{Written in Oct 20, 2011}
\end{flushright}

\paragraph{Minimum Variance Estimate.}

Ignoring any large time trend in the light curve, we can write any
realization as 
\begin{equation} 
\by = \bs + \bn
\end{equation}
and define
\begin{equation} 
\bcs \equiv \left<\bs\bs^T\right> ,\; \bcn \equiv
\left<\bn\bn^T\right>,\; \bcsstar \equiv \left<(\sstar)\bs\right>
\end{equation}
We then have the minimum variance estimate~(MVE) $\sstarhat$ for
$\sstar$ as
\begin{equation} 
\sstarhat = \bcsstar^T[\bcs+\bcn]^{-1}\by
\label{eqn:mve}
\end{equation}
with the variance of $\sstar$ about $\sstarhat$ as
\begin{equation} 
\left<(\sstar - \sstarhat)^2\right> = \left<\sstar^2\right> - \bcsstar^T[\bcs+\bcn]^{-1}\bcsstar
\label{eqn:variance}
\end{equation}

\paragraph{Unconstrained Realizations of the Underlying Process.} 

Given the symmetric, postive-definite covariance matrix $\bcc \equiv \bcs+\bcn$,
generating a random realization is straightforward.
\begin{itemize}
\item Cholesky decompose $\bcc = MM^T$, where $M$ is the lower
triangular matrix.
\item Generate $\br \sim \mathcal{N}(\mathbf{0}, I)$
\item Compute $\bx = M\br$ so that
$\left<\bx\bx^T\right>=\left<M\br (M\br)^T
\right>=\left<M\br\br^TM^T\right>=\bcc$
\end{itemize}

\paragraph{Predictive Distribution from the Observed Light Curve.}

Given $\bcc$ and the observed light curve vector $\by$, we can
compute Equation~(\ref{eqn:mve}) and~(\ref{eqn:variance}) by
\begin{itemize}
\item Cholesky decompose $\bcc = MM^T$, where $M$ is the lower.
\item Solve $MA=\by$ for $A$ and then $M^T\balpha=A$ for
$\balpha$, so that $\bcc\balpha=(MM^T)\balpha=\by$, thus
$\balpha=\bcc^{-1}\by$.
\item $\sstarhat = \bcsstar^T\balpha$ is the MVE.
\item Solve $M\bv=\bcsstar$ for $\bv$
\item $\left<(\sstar - \sstarhat)^2\right> = \left<\sstar^2\right> -
\bv^T\bv$, given $\bcsstar^T\bcc^{-1}\bcsstar =
\bcsstar^T(M^T)^{-1}M^{-1}\bcsstar = \bv^T\bv$
\end{itemize}

\paragraph{Constrained Realizations of the Underlying Process by
Observed Light Curves.}
A typical realization contrained by the data at unmeasured epoch can
be produced as $\sstar$ by
\begin{equation} 
\sstar = \ustar + \sstarhat
\end{equation}
where $\ustar$ is a Gaussian process with zero mean and correlation matrix $\bcq$
\begin{equation} 
\bcq = [\bcs^{-1}+\bcn^{-1}]^{1} = \bcs[\bcs+\bcn]^{-1}\bcn =
\bcn[\bcs+\bcn]^{-1}\bcs
\end{equation}
$\ustar$ can be generated using the same way as in the unconstrained
prediction case using the Cholesky decomposition of $\bcq$.



\bibliography{/home/nye/usr/bib/library}
\bibliographystyle{apj}


\end{document}
